\Introduction
 
В настоящее время в вузах существует практика сбора сведений о публикациях преподавательского состава и студентов кафедр. Эти сведения подаются преподавателями коллективно или в индивидуальном порядке ответственному лицу, которое на основе полученных данных формирует единый отчет в виде, требуемом отделом научно-технической информации.

Формирование отчета представляет собой рутинный труд по переносу предоставленных данных в сводную таблицу, отвечающую заявленным требованиям. Как и во всякой рутинной работе, выполняемой вручную, велико влияние человеческого фактора на результат, будь то опечатки, в том числе вставка информации не в ту ячейку, или пропущенные по неосторожности публикации. Описанные ошибки ответственного лица существенно снижают эффективность и качество работы, а обнаружение этих ошибок не всегда является простой задачей.

Помимо этого, требования к оформлению итогового отчета могут изменяться, поэтому не всегда очевидно, какая информация может пригодиться, а какая - нет. В зависимости от характера изменений, необходимо будет переделать один или несколько столбцов сводной таблицы, вручную переформатировать некоторые ячейки, добавить или удалить определенную информацию. При большом количестве исходных данных любое изменение в требованиях к итоговому отчету может обернуться трудо- и времязатратными  исправлениями.

Очевидно, что ручная обработка большого объема библиографических данных является неэффективной как по времени, так и по качеству; с другой стороны, выполнение данной работы компьютером гарантирует отсутствие ошибок по невнимательности, универсальность обработки различных форматов данных, структуризацию и быстроту обработки и, как следствие, своевременную отчетность.

Таким образом, \textbf{является актуальной} задача выполнения процесса классификации библиографических данных с помощью компьютера, которая заключается в переводе этих данных из различных форматов в единое представление.
 
Существующие информационные системы, использующие алгоритмы обработки библиографических данных, предполагают либо строгое соответствие данных определенным стандартам, таким как ГОСТ 5.0.7-2008, MARC и другим, что на практике труднодостижимо, так как речь идет не о переводе из одного формата в другой, а о переносе подаваемых преподавательским составом в независимое представление; либо используют современные технологии в сочетании с англоязычными словарями, что, пусть и исключает привязку к определенному формату, существенно влияет на качество работы с русскоязычными библиографическими сведениями в целом и со сведениями, оформленными в соответствии с ГОСТ 5.0.7-2008 в частности.

Поэтому было принято решение о разработке собственного алгоритма, в полной мере поддерживающего русский язык и способного проявлять гибкость при обработке и не следовать определенному формату.

\textbf{Объектом исследования} являются библиографические данные в слабо структурированных форматах на основе текстового представления.

\textbf{Предметом исследования} из всей области научных изысканий, представляющих объект исследования, были выбраны алгоритмы классификации библиографических данных.

\textbf{Цель работы} заключается в повышении быстродействия процесса обработки и структуризации библиографических данных в высших учебных заведениях и других организациях, связанных с научной деятельность.

\textbf{Задачами исследования} являются:
\begin{itemize}
	\item аналитический обзор современного состояния методов и средств для классификации библиографических данных (КБД);
	\item формализация задачи КБД;
	\item разработка методики КБД;
	\item разработка алгоритма КБД;
	\item программная реализация разработанных алгоритма и методики КБД;
	\item оценка эффективности разработанных алгоритма и методики КБД.
\end{itemize}

В диссертации в качестве \textbf{методов исследования} используются методы теории автоматов, методы построения нейронных сетей, статистические методы и методы машинного обучения.

\textbf{Научная новизна исследования} состоит в создании новых методики и алгоритма классификации библиографических данных.

Были исследованы современные методы классификации библиографических данных. На основе этого анализа был выбран метод условно-случайных полей, наиболее подходящий для решения проблемы компьютеризации процесса классификации библиографических данных.

Выбранный метод условно-случайных полей был применен к задаче классификации библиографических данных на русском языке, оформленных в разной степени соответствия стандарту ГОСТ 5.0.7-2008, наиболее распространенному в нашей стране.

Для проверки работы метода был выбран набор меток, отражающих различные структурные элементы библиографической записи, составлена графовая модель состояний и переходов, а также проведена работа по выделению ключевых признаков структурных элементов и составлению их функций.

Среди ключевых функций признаков были выделены следующие группы: функции положения слова в предложении, функции символьного состава слова, регистровые функции и функции совпадения со служебными символами стандарта ГОСТ 5.0.7-2008. Ряд функций признаков был отброшен в процессе исследования, так как отрицательно сказывался на точности работы метода.

На основе полученных результатов были сформированы методика и алгоритм классификации библиографических данных, а также было разработано программное средство классификации библиографических данных, реализующее метод условно-случайных полей в соответствии с разработанной графовой моделью состояний и переходов и выделенными функциями признаков.

\textbf{Практическая значимость результатов} исследования заключается в обеспечении эффективного управления библиографическими сведениями вузов и других образовательных и научных учреждений, библиотек, научно-исследовательских и прочих организаций, работающих с библиографическими данными, за счет замены ручной обработки данных на компьютерную, осуществляемую на основе разработанных методики и алгоритма.

Результаты исследования позволят выполнять обработку и структуризацию библиографических данных на русском языке без использования явно заданного формата, опираясь только на признаки отдельных структурных компонентов библиографической записи, а не на их последовательность или структуру записи в целом.

Такой подход к обработке позволит осуществлять перевод библиографических данных из одного представления или формата в другой с минимальным участием оператора, в том числе исправлять ошибки форматирования и следования структурных компонентов в рамках одного стандарта оформления.

\textbf{Структура диссертации} состоит из введения, четырех глав, заключения, списка литературы и приложений. Работы содержит \_ страниц основного текста, \_ страниц с рисунками и таблицами, список литературы из \_ наименований, \_ приложения на \_ страницах.

\textbf{На защиту выносятся} следующие положения и результаты:
\begin{itemize}
	\item Результаты аналитического обзора современного состояния методов и средств для классификации библиографических данных (КБД);
	\item Формализованное представление задачи КБД;
	\item Методика КБД на основе разработанного алгоритма;
	\item Алгоритм КБД на основе метода условно-случайных полей;
	\item Программная реализация разработанных алгоритма и методики КБД;
	\item Результаты оценки эффективности разработанных алгоритма и методики КБД.
\end{itemize}
