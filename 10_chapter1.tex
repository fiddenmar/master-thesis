\chapter{Обзор современного состояния информационных систем для обработки и структуризации библиографических данных. Постановка задачи диссертационных исследований}

В настоящее время обработка библиографических данных для учета научных трудов в высших учебных заведениях производится вручную, что существенно сказывается на временных и трудовых затратах сотрудников. Эти сведения подаются преподавателями коллективно или в индивидуальном порядке ответственному лицу, которое на основе полученных данных формирует единый отчет в виде, требуемом отделом научно-технической информации. 

Формирование отчета представляет собой рутинный труд по переносу предоставленных данных в сводную таблицу, отвечающую заявленным требованиям. Как и во всякой рутинной работе, выполняемой вручную, качество выполняемой работы сильно страдает из-за человеческого фактора: ошибок, опечаток и невнимательности.

Помимо этого, требования к оформлению итогового отчета могут изменяться. В зависимости от характера изменений, необходимо будет переделать один или несколько столбцов сводной таблицы, вручную переформатировать некоторые ячейки, добавить или удалить определенную информацию. При большом количестве исходных данных любое изменение в требованиях к итоговому отчету может обернуться трудо- и времязатратными  исправлениями.

В то же время компьютерная обработка данных с минимальным участием человека лишена этих недостатков и может выполняться в кратчайшие сроки. Выполнение данной работы компьютером гарантирует отсутствие ошибок по невнимательности, структуризацию и быстроту обработки и, как следствие, своевременную отчетность.

Область применения задачи обработки и классификации библиографических данных не ограничивается высшими учебными заведениями и учетом научных трудов. Данная задача является важной составляющей электронно-библиотечных систем, систем управления библиографической информацией, обрабатывающих модулей взаимодействия с библиографическими базами данных, а так же других информационных систем, тесно связанных с обработкой и структуризацией библиографических данных.

\section{Принципы построения информационных систем для обработки и структуризации библиографических данных}

Информационная система - это система, предназначенная для хранения, поиска и обработки информации, и соответствующие организаторские ресурсы, такие как человеческие, технические, финансовые и другие, обеспечивающие и распространяющие эту информацию.

Информационная система предназначена для своевременного обеспечения надлежащих людей надлежащей информацией, то есть для удовлетворения конкретных информационных потребностей в рамках определенной предметной области, при этом результатом функционирования информационных систем является информационная продукция — документы, информационные массивы, базы данных и информационные услуги.

Достаточно широкое понимание информационной системы подразумевает, что её неотъемлемыми компонентами являются данные, техническое и программное обеспечение, а также персонал и организационные мероприятия. Широко трактует понятие «информационной системы» федеральный закон Российской Федерации «Об информации, информационных технологиях и о защите информации», подразумевая под информационной системой совокупность содержащейся в базах данных информации и обеспечивающих её обработку информационных технологий и технических средств.

Среди российских ученых в области информатики наиболее широкое определение ИС дает М. Р. Когаловский \cite{Kogalovskij2013}, по мнению которого в понятие информационной системы помимо данных, программ, аппаратного обеспечения и людских ресурсов следует также включать коммуникационное оборудование, лингвистические средства и информационные ресурсы, которые в совокупности образуют систему, обеспечивающую «поддержку динамической информационной модели некоторой части реального мира для удовлетворения информационных потребностей пользователей».

Более узкое понимание информационной системы ограничивает её состав данными, программами и аппаратным обеспечением. Интеграция этих компонентов позволяет автоматизировать процессы управления информацией и целенаправленной деятельности конечных пользователей, направленной на получение, модификацию и хранение информации. Так, российский стандарт ГОСТ РВ 51987 подразумевает под ИС «автоматизированную систему, результатом функционирования которой является представление выходной информации для последующего использования». ГОСТ Р 53622-2009 использует термин информационно-вычислительная система для обозначения совокупности данных (или баз данных), систем управления базами данных и прикладных программ, функционирующих на вычислительных средствах как единое целое для решения определенных задач.

В деятельности организации информационная система рассматривается как программное обеспечение, реализующее деловую стратегию организации. При этом хорошей практикой является создание и развертывание единой корпоративной информационной системы, удовлетворяющей информационные потребности всех сотрудников, служб и подразделений организации. Однако на практике создание такой всеобъемлющей информационной системы слишком затруднено или даже невозможно, вследствие чего на предприятии обычно функционируют несколько различных систем, решающих отдельные группы задач: управление производством, финансово-хозяйственная деятельность, электронный документооборот и т. д. Часть задач бывает покрыта одновременно несколькими информационными системами, часть задач — вовсе не автоматизирована. Такая ситуация получила название «лоскутной автоматизации» и является довольно типичной для многих предприятий.

По степени распределенности различают настольные, также именуемые локальными, в которых все компоненты, такие как база данных, система управления базой данных и клиентские приложения, находятся на одном компьютере; распределенные, в которых перечисленные компоненты располагаются на нескольких компьютерах. Распределенные информационные системы разделяют на файл-серверные ИС и клиент-серверные ИС, различающиеся архитектурами "файл-сервер" и "клиент-сервер" соответственно.

В файл-серверных ИС базу данных размещают на файловом сервере, а СУБД и клиентские приложения размещают на рабочих станциях.

Про файл-серверные архитектуры.

В клиент-серверных ИС и БД, и СУБД находятся на сервере, а то время как на рабочих станциях находятся только клиентские приложения.

Про клиент-серверные архитектуры.

Клиент-серверные ИС бывают двухзвенные и многозвенные.

Двухзвенные ИС используют два типа "звеньев": сервер базы данных, на которых находятся БД и СУБД, и рабочие станции, на которых находятся клиентские приложения, обращающиеся к СУБД напрямую.

В многозвенных ИС добавляют несколько промежуточных "звеньев": серверы приложений, которые являются прослойкой между клиентским и серверным функционалом. Клиентские приложения не обращаются к СУБД напрямую, а взаимодействуют с промежуточными звеньями.

По степени автоматизации ИС разделяют на автоматические, в которых автоматизация является полной, вмешательство персонала не требуется или требуется только периодически; и автоматизированные, в которых автоматизация не является полной и требуется постоянное вмешательство персонала.

По характеру обработки данных ИС делятся на информационно-справочные, или информационно-поисковые ИС, в которых нет сложных алгоритмов обработки данных, а целью системы является поиск и выдача информации в удобном виде; и решающие, в которых данные подвергаются обработке по сложным алгоритмам. К таким системам в первую очередь относят автоматизированные системы управления и системы поддержки принятия решений.

По охвату задач ИС разделяют на персональные ИС, предназначенные для решения некоторого круга задач одного человека; групповые ИС, ориентированные на коллективное использование информации членами рабочей группы или подразделения; и корпоративные ИС, автоматизирующие бизнес-процессы организации, обеспечивая информационную согласованность и прозрачность.

Информационная система для обработки и структуризации библиографических данных может попадать под одну или несколько классификаций в зависимости от масштаба системы и численности персонала.

В том случае, если есть ответственно лицо, отвечающее за обработку библиографических данных в организации, к примеру - в высшем учебном заведении, ИС должна быть локальной, автоматизированной, информационно-поисковой или решающей и персональной.

В том случае, если ИС пользуется группа лиц, ИС должна быть распределенной, клиент-серверной, двухзвенной или многозвенной, автоматизированная и групповая или корпоративная.

В данной работе под информационной системой для обработки и структуризации библиографических данных будет подразумеваться локальная, клиент-серверная, автоматизированная, решающая и персональная ИС с одним оператором. 

Выделяют следующие основополагающие принципы, на которые следует опираться в процессе создания ИС:

\begin{itemize}
	\item принцип системности;
	\item принцип открытости;
	\item принцип совместимости;
	\item принцип унификации;
	\item принцип эффективности.
\end{itemize}

Принцип системности предполагает учет всех взаимосвязей, анализ отдельных частей системы как ее самостоятельных структурных составляющих и, параллельно, выявление роли каждой из них в функционировании всей системы в целом. Таким образом, реализуются процессы анализа и синтеза, фундаментальный смысл которых - разложение целого на составные части и воссоединение целого из частей. Данный принцип заключается в том, что должны быть установлены связи между структурными компонентами системы, обеспечивающие их взаимодействие при комплексном решении задачи.

Принцип открытости заключается в том, что внесенные в систему изменения, обусловленные различными причинами, осуществляется только путем дополнения системы без нарушения ее функционирования. На практике данный принцип трудно реализуем, так как требует глубокого анализа на предпроектном этапе работ, который подразумевает разделение решаемых задач на подгруппы и прогнозирование возможных направлений решения выбранных групп задач. Отдельной сложностью при реализации данного метода является общая непостоянность в компьютерном мире: постоянно развивается оборудование, системы связи и протоколы передачи данных.

Принцип совместимости подразумевает реализацию информационных интерфейсов таким образом, что они могут взаимодействовать с другими ИС согласно установленным правилам. В настоящее время это особенно актуально для сетевых связей на локальном и глобальном уровнях. Однако, если в локальных сетях несложно установить и соблюдать стандарты взаимодействия, то выход в глобальные сети требует дополнительных мер по защите инфорамции, соблюдения стандартов информационных обменов и корректности составления сообщений.

Принцип унификации заключается в рациональном использовании типовых и унифицированных элементов, проектных решений и прикладных программ. Таким образом, необходимо разрабатывать задачи таким образом, чтобы они подходили к наиболее широкому кругу объектов. 

Принцип эффективности предусматривает достижение рационалного соотношения между затратами на создание ИС и конечными результатами, отражающими прибыльность и внедрение автоматизации в управленческие процессы.

Помимо описанных выше принципов построения ИС, следует выделить еще один: дружественность к пользователю. Развитые интерфейсы в процессах взаимодействия в системе "человек-машина" позволяют эффективно работать пользователю, не имеющему специальной компьютерной подготовки.

Важной частью информационной системы обработки и структуризации библиографической данных является база данных, хранящая результаты работы обрабатывающего и структурирующего модулей.

База данных - именованная совокупность данных, наотражающая состояние объектов и их отношений в рассматриваемой предметной области. Под предметной областью понимается некоторая область человеческой деятельности или область реального мира, на основе которой создается БД и ее структура.

В данном исследовании в качестве предметной области, на основе которой составляется БД и ее структура, рассматриваются библиографические данные.

Современные БД должны следовать следующим основным принципам:
\begin{itemize}
	\item высокое быстродействие, т.е. малый промежуток времени с момента запроса к БД до фактического получения данных;
	\item простота обновления данных;
	\item независимость данных, т.е. возможность изменения логической и физической структуры БД без изменения представлений пользователей;
	\item совместное использование анных многими пользователями;
	\item безопасность данных;
	\item адекватность отображения данных в соответствии с предметной областью;
	\item простой пользовательский интерфейс.
\end{itemize}

Основные задачи проектирования баз данных:
\begin{itemize}
	\item обеспечение хранения в БД всей необходимой информации;
	\item обеспечение возможности получения данных по всем необходимым запросам;
	\item сокращение избыточности и дублирования данных;
	\item обеспечение целостности данных.
\end{itemize}

Основные этапы проектирования баз данных:
\begin{itemize}
	\item инфологическое проектирование - построение формализованной модели модели предметной области;
	\item даталогическое проектирование - отобраение инфологической модели на модель данных, используемую в конкретной СУБД.
	\item физическое проектирование - реализация даталогической модели средствами конкретной СУБД.
\end{itemize}

Под библиографичекой базой данных, необходимой для функционирования ИС для обработки и структуризации библиографических данных, понимают такую базу данных, в которой хранится полная информация о типах, авторах, названиях, издательствах, и других структурных элементов библиографических записей.

\section{Основные функциональные возможности информационных систем для обработки и структуризации библиографических данных}

С точки зрения функциональных возможностей ИС для обработки и структуризации библиографических данных можно рассматривать как систему управления библиографической информацией.

Системы управления библиографической информацией — это системы, позволяющие исследователям, учёным и писателям создавать и повторно использовать библиографические ссылки. После того как ссылка создана, она используется для создания библиографии, то есть списка библиографических ссылок, в научной статье, монографии, книге.

Программное обеспечение обычно включает базу данных и систему генерации отобранных ссылок в форматах, требуемых научными журналами. Современные системы управления библиографической информацией могут быть интегрированы с текстовыми процессорами таким образом, что список ссылок создаётся автоматически и добавляется в документ. Это избавляет от неприятности забыть включить ссылку на источник, цитируемый в тексте. Также системы имеют возможность импортировать и экспортировать детали библиографических ссылок из библиографических баз данных.

С развитием Интернета, появились онлайновые системы управления библиографической информацией, предоставляющие доступ с любого компьютера.

Следует различать системы управления библиографической информацией и библиографические базы данных, которые пытаются собрать данные о всех статьях по данной дисциплине или группе дисциплин, например, Medline, ISI Web of Knowledge или Scopus. Это большие базы данных, установленные на специальных серверах. В системах управления библиографической информацией создаются значительно меньшие базы публикаций, используемые одним человеком или группой людей; такие базы можно легко установить на отдельном персональном компьютере.

К основным функциональным возможностям ИС для обработки и структуризации библиографических данных можно отнести следующее:
\begin{itemize}
	\item прием библиографических данных в одном из допустимых форматов;
	\item предобработка библиографических данных и вычленение отдельных библиографических записей;
	\item обработка библиографических записей и выделение из них структурных элементов;
	\item постобработка структурных элементов и их внесение в БД;
	\item извлечение структурных элементов из БД и формирование библиографических записей в требуемом формате.
\end{itemize}

Основное внимание данного исследования будет сконцетрировано на третьей функциональной возможности: обработке библиографических записей и выделению из них структурных элементов.

\section{Структура библиографических данных}

Библиографическая информация - по определенным правилам организованная информация о документах, содействующая реализации соответствий между документами и их потребителями.

Выделяют три функции библиографической информации в системе "документ - потребитель": поисковую, коммуникативную и оценочную.

Одним из видов представления библиографической информации является библиографическая запись.

Библиографическая запись — наименьшая единица библиографического списка, состоящая из заголовка и библиографического описания, одна из форм библиографической информации. Используется для идентификации документа и осуществления библиографического поиска.

Библиографическая ссылка - один из видов библиографических записей, в России регулируется стандартом ГОСТ 7.0.5-2008.

Именно библиографические ссылки, как наиболее часто употребляемый вид библиографических записей, будут рассматриваться в данном ииследовании.

\subsection{Структурные элементы библиографических данных}
Библиографическое описание на книгу или любой другой документ составляется по определенным правилам. Оно содержит библиографические сведения о документе, приведенные в определенном порядке, позволяющие идентифицировать документ и дать его общую характеристику.

Библиографическое описание является основной частью библиографической записи. Запись также может включать: заголовок, классификационные индексы, предметные рубрики, аннотацию или реферат, справки о связях с другими записями, шифры хранения, дату завершения библиографической обработки издания и другие сведения. Степень полноты записи определяется целями и задачами ее составителя. Как правило, для целей создания списка использованных источников (литературы в научной рукописи) используется неполная схема описания, а только ее обязательная часть. Полные и обязательные элементы описания, а также порядок расположения и разделяющие знаки стандартизированы государственным стандартом ГОСТ 7.1-2003.

В зависимости от структуры описания различают:
\begin{itemize}
	\item одноуровневое библиографическое описание - это описание одного отдельно взятого (одночастного) документа (монографии, учебника, справочника, сборника статей, архивного документа и т.д.);
	\item многоуровневое библиографическое описание - это описание многочастного документа (многотомное издание);
	\item аналитическое библиографическое описание - это описание части документа (статья из периодического издания или сборника).
\end{itemize}

Краткая схема библиографического описания (описание состоит из обязательных элементов) схематично может быть представлена так:

Заголовок описания. Основное заглавие: сведения, относящиеся к заглавию / Сведения об ответственности. - Сведения об издании. - Выходные данные. - Объем.

Заголовок - это элемент библиографической записи, расположенный перед основным заглавием произведения.

Он может включать имя лица (имя лица - условно применяемое понятие, включающее фамилию, инициалы или имя и отчество, псевдоним, личное имя или прозвище в качестве фамилии), наименование организации, унифицированное заглавие произведения, обозначение документа, географическое название, иные сведения. Заголовок применяют при составлении записи на произведение одного, двух и трех авторов. Если авторов четыре и более, то заголовок не применяют, запись составляют под заглавием произведения.

При наличии двух и трех авторов указывают только имя первого автора или выделенного на книге каким-либо способом (цветом, шрифтом). Имена всех авторов приводят в библиографическом описании в сведениях об ответственности.

Основным заглавием является заглавие книги или статьи, а сведением, относящимся к заглавию - пояснение жанра, типа издания, например, сборник статей, учебное пособие и т.п.

Сведения об ответственности - это сведения о соавторах, переводчиках, редакторах и/или о той организации, которая принимает на себя ответственности за данную публикацию.

Сведения об издании включают качественную и количественную характеристику документа - переработанное, стереотипное, 2-е и т. п.

Выходные данные - это наименование города, издательства, где опубликована книга и года издания. Москва, Ленинград, Санкт-Петербург, Лондон, Париж и Нью-Йорк сокращаются (М., Л., СПб., L., P., N-Y.). Все остальные города пишутся полностью (Новосибирск, Киев). Названия издательств сокращаются в соответствии с ГОСТом. Названия издательств книг, опубликованных до 1917 года, пишутся полностью. Дата для книги означает год издания.

Объем - это количество страниц или страницы, на которых опубликована статья в журнале или сборнике.

В данном исследовании основное внимание сконцетрировано на обработке и структуризации затекстовых библиографических ссылок, оформление которых реглируется стандартом ГОСТ 5.0.7 - 2008 и имеет следующий общий вид:
\begin{itemize}
	\item заголовок;
	\item основное заглавие документа;
	\item общее обозначение материала;
	\item сведения, относящиеся к заглавию;
	\item сведения об ответственности;
	\item сведения об издании;
	\item выходные данные;
	\item физическую характеристику документа;
	\item сведения о местоположении объекта ссылки в документе (если ссылка на часть документа);
	\item сведения о серии;
	\item обозначение и порядковый номер тома или выпуска (для ссылок на публикации в многочастных или сериальных документах);
	\item сведения о документе, в котором опубликован объект ссылки;
	\item примечания;
	\item Международный стандартный номер.
\end{itemize}

\subsection{Стандарты представления библиографических данных}


\subsubsection{ГОСТ 7.0.5 - 2008}


\subsubsection{АПА}


\subsubsection{МЛА}


\subsection{Машиночитаемые форматы библиографических данных}
1234

\section{Анализ основных параметров, достоинств и недостатков современных информационных систем для обработки и структуризации библиографических данных}
1234

\section*{Выводы к главе 1}
\addcontentsline{toc}{section}{Выводы к главе 1}
1234
