\chapter{Обзор современного состояния информационных систем для обработки и структуризации библиографических данных. Постановка задачи диссертационных исследований}

В настоящее время обработка библиографических данных для учета научных трудов в высших учебных заведениях производится вручную, что существенно сказывается на временных и трудовых затратах сотрудников. Эти сведения подаются преподавателями коллективно или в индивидуальном порядке ответственному лицу, которое на основе полученных данных формирует единый отчет в виде, требуемом отделом научно-технической информации. 

Формирование отчета представляет собой рутинный труд по переносу предоставленных данных в сводную таблицу, отвечающую заявленным требованиям. Как и во всякой рутинной работе, выполняемой вручную, качество выполняемой работы сильно страдает из-за человеческого фактора: ошибок, опечаток и невнимательности.

Помимо этого, требования к оформлению итогового отчета могут изменяться. В зависимости от характера изменений, необходимо будет переделать один или несколько столбцов сводной таблицы, вручную переформатировать некоторые ячейки, добавить или удалить определенную информацию. При большом количестве исходных данных любое изменение в требованиях к итоговому отчету может обернуться трудо- и времязатратными  исправлениями.

В то же время компьютерная обработка данных с минимальным участием человека лишена этих недостатков и может выполняться в кратчайшие сроки. Выполнение данной работы компьютером гарантирует отсутствие ошибок по невнимательности, структуризацию и быстроту обработки и, как следствие, своевременную отчетность.

Область применения задачи обработки и классификации библиографических данных не ограничивается высшими учебными заведениями и учетом научных трудов. Данная задача является важной составляющей электронно-библиотечных систем, систем управления библиографической информацией, обрабатывающих модулей взаимодействия с библиографическими базами данных, а так же других информационных систем, тесно связанных с обработкой и структуризацией библиографических данных.

\section{Принципы построения информационных систем для обработки и структуризации библиографических данных}

Информационная система () - это система, предназначенная для хранения, поиска и обработки информации, и соответствующие организаторские ресурсы, такие как человеческие, технические, финансовые и другие, обеспечивающие и распространяющие эту информацию.

ИС предназначена для своевременного обеспечения надлежащих людей надлежащей информацией, то есть для удовлетворения конкретных информационных потребностей в рамках определенной предметной области, при этом результатом функционирования информационных систем является информационная продукция — документы, информационные массивы, базы данных и информационные услуги.

Достаточно широкое понимание информационной системы подразумевает, что её неотъемлемыми компонентами являются данные, техническое и программное обеспечение, а также персонал и организационные мероприятия. Широко трактует понятие «информационной системы» федеральный закон Российской Федерации «Об информации, информационных технологиях и о защите информации», подразумевая под информационной системой совокупность содержащейся в базах данных информации и обеспечивающих её обработку информационных технологий и технических средств.

Среди российских ученых в области информатики наиболее широкое определение ИС дает М. Р. Когаловский \cite{Kogalovskij2013}, по мнению которого в понятие информационной системы помимо данных, программ, аппаратного обеспечения и людских ресурсов следует также включать коммуникационное оборудование, лингвистические средства и информационные ресурсы, которые в совокупности образуют систему, обеспечивающую «поддержку динамической информационной модели некоторой части реального мира для удовлетворения информационных потребностей пользователей».

Более узкое понимание информационной системы ограничивает её состав данными, программами и аппаратным обеспечением. Интеграция этих компонентов позволяет автоматизировать процессы управления информацией и целенаправленной деятельности конечных пользователей, направленной на получение, модификацию и хранение информации. Так, российский стандарт ГОСТ РВ 51987 подразумевает под ИС «автоматизированную систему, результатом функционирования которой является представление выходной информации для последующего использования». ГОСТ Р 53622-2009 использует термин информационно-вычислительная система для обозначения совокупности данных (или баз данных), систем управления базами данных и прикладных программ, функционирующих на вычислительных средствах как единое целое для решения определенных задач.

В деятельности организации информационная система рассматривается как программное обеспечение, реализующее деловую стратегию организации. При этом хорошей практикой является создание и развертывание единой корпоративной информационной системы, удовлетворяющей информационные потребности всех сотрудников, служб и подразделений организации. Однако на практике создание такой всеобъемлющей информационной системы слишком затруднено или даже невозможно, вследствие чего на предприятии обычно функционируют несколько различных систем, решающих отдельные группы задач: управление производством, финансово-хозяйственная деятельность, электронный документооборот и т. д. Часть задач бывает покрыта одновременно несколькими информационными системами, часть задач — вовсе не автоматизирована. Такая ситуация получила название «лоскутной автоматизации» и является довольно типичной для многих предприятий.

По степени распределенности различают настольные, также именуемые локальными, в которых все компоненты, такие как база данных, система управления базой данных и клиентские приложения, находятся на одном компьютере; распределенные, в которых перечисленные компоненты располагаются на нескольких компьютерах. Распределенные информационные системы разделяют на файл-серверные ИС и клиент-серверные ИС, различающиеся архитектурами «файл-сервер» и «клиент-сервер» соответственно.

В файл-серверных ИС базу данных размещают на файловом сервере, а СУБД и клиентские приложения размещают на рабочих станциях.

В клиент-серверных ИС и БД, и СУБД находятся на сервере, а то время как на рабочих станциях находятся только клиентские приложения.

Клиент-серверные ИС бывают двухзвенные и многозвенные.

Двухзвенные ИС используют два типа «звеньев»: сервер базы данных, на которых находятся БД и СУБД, и рабочие станции, на которых находятся клиентские приложения, обращающиеся к СУБД напрямую.

В многозвенных ИС добавляют несколько промежуточных «звеньев»: серверы приложений, которые являются прослойкой между клиентским и серверным функционалом. Клиентские приложения не обращаются к СУБД напрямую, а взаимодействуют с промежуточными звеньями.

По степени автоматизации ИС разделяют на автоматические, в которых автоматизация является полной, вмешательство персонала не требуется или требуется только периодически; и автоматизированные, в которых автоматизация не является полной и требуется постоянное вмешательство персонала.

По характеру обработки данных ИС делятся на информационно-справочные, или информационно-поисковые ИС, в которых нет сложных алгоритмов обработки данных, а целью системы является поиск и выдача информации в удобном виде; и решающие, в которых данные подвергаются обработке по сложным алгоритмам. К таким системам в первую очередь относят автоматизированные системы управления и системы поддержки принятия решений.

По охвату задач ИС разделяют на персональные ИС, предназначенные для решения некоторого круга задач одного человека; групповые ИС, ориентированные на коллективное использование информации членами рабочей группы или подразделения; и корпоративные ИС, автоматизирующие бизнес-процессы организации, обеспечивая информационную согласованность и прозрачность.

ИС для обработки и структуризации библиографических данных может попадать под одну или несколько классификаций в зависимости от масштаба системы и численности персонала.

В том случае, если иммется ответственно лицо, отвечающее за обработку библиографических данных в организации, к примеру - в высшем учебном заведении, ИС должна быть локальной, автоматизированной, информационно-поисковой или решающей и персональной.

В том случае, если ИС пользуется группа лиц, ИС должна быть распределенной, клиент-серверной, двухзвенной или многозвенной, автоматизированная и групповая или корпоративная.

В данной работе под информационной системой для обработки и структуризации библиографических данных будет подразумеваться локальная, клиент-серверная, автоматизированная, решающая и персональная ИС с одним оператором. 

Выделяют следующие основополагающие принципы, на которые следует опираться в процессе создания ИС:

\begin{itemize}
	\item принцип системности;
	\item принцип открытости;
	\item принцип совместимости;
	\item принцип унификации;
	\item принцип эффективности.
\end{itemize}

Принцип системности предполагает учет всех взаимосвязей, анализ отдельных частей системы как ее самостоятельных структурных составляющих и, параллельно, выявление роли каждой из них в функционировании всей системы в целом. Таким образом, реализуются процессы анализа и синтеза, фундаментальный смысл которых - разложение целого на составные части и воссоединение целого из частей. Данный принцип заключается в том, что должны быть установлены связи между структурными компонентами системы, обеспечивающие их взаимодействие при комплексном решении задачи.

Принцип открытости заключается в том, что внесенные в систему изменения, обусловленные различными причинами, осуществляется только путем дополнения системы без нарушения ее функционирования. На практике данный принцип трудно реализуем, так как требует глубокого анализа на предпроектном этапе работ, который подразумевает разделение решаемых задач на подгруппы и прогнозирование возможных направлений решения выбранных групп задач. Отдельной сложностью при реализации данного метода является общая непостоянность в компьютерном мире: постоянно развивается оборудование, системы связи и протоколы передачи данных.

Принцип совместимости подразумевает реализацию информационных интерфейсов таким образом, что они могут взаимодействовать с другими ИС согласно установленным правилам. В настоящее время это особенно актуально для сетевых связей на локальном и глобальном уровнях. Однако, если в локальных сетях несложно установить и соблюдать стандарты взаимодействия, то выход в глобальные сети требует дополнительных мер по защите инфорамции, соблюдения стандартов информационных обменов и корректности составления сообщений.

Принцип унификации заключается в рациональном использовании типовых и унифицированных элементов, проектных решений и прикладных программ. Таким образом, необходимо разрабатывать задачи таким образом, чтобы они подходили к наиболее широкому кругу объектов. 

Принцип эффективности предусматривает достижение рационалного соотношения между затратами на создание ИС и конечными результатами, отражающими прибыльность и внедрение автоматизации в управленческие процессы.

Помимо описанных выше принципов построения ИС, следует выделить еще один: дружественность к пользователю. Развитые интерфейсы в процессах взаимодействия в системе «человек-машина» позволяют эффективно работать пользователю, не имеющему специальной компьютерной подготовки.

Важной частью информационной системы обработки и структуризации библиографической данных является база данных, хранящая результаты работы обрабатывающего и структурирующего модулей.

База данных - именованная совокупность данных, наотражающая состояние объектов и их отношений в рассматриваемой предметной области. Под предметной областью понимается некоторая область человеческой деятельности или область реального мира, на основе которой создается БД и ее структура.

В данном исследовании в качестве предметной области, на основе которой составляется БД и ее структура, рассматриваются библиографические данные.

Современные БД должны следовать следующим основным принципам:
\begin{itemize}
	\item высокое быстродействие, т.е. малый промежуток времени с момента запроса к БД до фактического получения данных;
	\item простота обновления данных;
	\item независимость данных, т.е. возможность изменения логической и физической структуры БД без изменения представлений пользователей;
	\item совместное использование данных многими пользователями;
	\item безопасность данных;
	\item адекватность отображения данных в соответствии с предметной областью;
	\item простой пользовательский интерфейс.
\end{itemize}

Основные задачи проектирования баз данных:
\begin{itemize}
	\item обеспечение хранения в БД всей необходимой информации;
	\item обеспечение возможности получения данных по всем необходимым запросам;
	\item сокращение избыточности и дублирования данных;
	\item обеспечение целостности данных.
\end{itemize}

Основные этапы проектирования баз данных:
\begin{itemize}
	\item инфологическое проектирование - построение формализованной модели модели предметной области;
	\item даталогическое проектирование - отобраение инфологической модели на модель данных, используемую в конкретной СУБД.
	\item физическое проектирование - реализация даталогической модели средствами конкретной СУБД.
\end{itemize}

Под библиографичекой базой данных, необходимой для функционирования ИС для обработки и структуризации библиографических данных, понимают такую базу данных, в которой хранится полная информация о типах, авторах, названиях, издательствах, и других структурных элементов библиографических записей.

\section{Основные функциональные возможности информационных систем для обработки и структуризации библиографических данных}

С точки зрения функциональных возможностей ИС для обработки и структуризации библиографических данных можно рассматривать как систему управления библиографической информацией.

Системы управления библиографической информацией — это системы, позволяющие исследователям, учёным и писателям создавать и повторно использовать библиографические ссылки. После того как ссылка создана, она используется для создания библиографии, то есть списка библиографических ссылок, в научной статье, монографии, книге.

Программное обеспечение обычно включает базу данных и систему генерации отобранных ссылок в форматах, требуемых научными журналами. Современные системы управления библиографической информацией могут быть интегрированы с текстовыми процессорами таким образом, что список ссылок создаётся автоматически и добавляется в документ. Это избавляет от неприятности забыть включить ссылку на источник, цитируемый в тексте. Также системы имеют возможность импортировать и экспортировать детали библиографических ссылок из библиографических баз данных.

С развитием Интернета, появились онлайн-системы управления библиографической информацией, предоставляющие доступ с любого компьютера.

Следует различать системы управления библиографической информацией и библиографические базы данных, которые пытаются собрать данные о всех статьях по данной дисциплине или группе дисциплин, например, Medline, ISI Web of Knowledge или Scopus. Это большие базы данных, установленные на специальных серверах. В системах управления библиографической информацией создаются значительно меньшие базы публикаций, используемые одним человеком или группой людей; такие базы можно легко установить на отдельном персональном компьютере.

К основным функциональным возможностям ИС для обработки и структуризации библиографических данных можно отнести следующее:
\begin{itemize}
	\item прием библиографических данных в одном из допустимых форматов;
	\item предобработка библиографических данных и вычленение отдельных библиографических записей;
	\item обработка библиографических записей и выделение из них структурных элементов;
	\item постобработка структурных элементов и их внесение в БД;
	\item извлечение структурных элементов из БД и формирование библиографических записей в требуемом формате.
\end{itemize}

Основное внимание данного исследования будет сконцетрировано на третьей функциональной возможности: обработке библиографических записей и выделению из них структурных элементов.

\section{Структура библиографических данных}

Библиографическая информация - по определенным правилам организованная информация о документах, содействующая реализации соответствий между документами и их потребителями.

Выделяют три функции библиографической информации в системе «документ - потребитель»: поисковую, коммуникативную и оценочную.

Одним из видов представления библиографической информации является библиографическая запись.

Библиографическая запись — наименьшая единица библиографического списка, состоящая из заголовка и библиографического описания, одна из форм библиографической информации. Используется для идентификации документа и осуществления библиографического поиска.

Библиографическая ссылка - один из видов библиографических записей, в России регулируется стандартом ГОСТ 7.0.5-2008.

Именно библиографические ссылки, как наиболее часто употребляемый вид библиографических записей, будут рассматриваться в данном ииследовании.

\subsection{Структурные элементы библиографических данных}
Библиографическое описание на книгу или любой другой документ составляется по определенным правилам. Оно содержит библиографические сведения о документе, приведенные в определенном порядке, позволяющие идентифицировать документ и дать его общую характеристику.

Библиографическое описание является основной частью библиографической записи. Запись также может включать: заголовок, классификационные индексы, предметные рубрики, аннотацию или реферат, справки о связях с другими записями, шифры хранения, дату завершения библиографической обработки издания и другие сведения. Степень полноты записи определяется целями и задачами ее составителя. Как правило, для целей создания списка использованных источников (литературы в научной рукописи) используется неполная схема описания, а только ее обязательная часть. Полные и обязательные элементы описания, а также порядок расположения и разделяющие знаки стандартизированы государственным стандартом ГОСТ 7.1-2003.

В зависимости от структуры описания различают:
\begin{itemize}
	\item одноуровневое библиографическое описание - это описание одного отдельно взятого (одночастного) документа (монографии, учебника, справочника, сборника статей, архивного документа и т.д.);
	\item многоуровневое библиографическое описание - это описание многочастного документа (многотомное издание);
	\item аналитическое библиографическое описание - это описание части документа (статья из периодического издания или сборника).
\end{itemize}

Краткая схема библиографического описания (описание состоит из обязательных элементов) схематично может быть представлена так:

Заголовок описания. Основное заглавие: сведения, относящиеся к заглавию / Сведения об ответственности. - Сведения об издании. - Выходные данные. - Объем.

Заголовок - это элемент библиографической записи, расположенный перед основным заглавием произведения.

Он может включать имя лица (имя лица - условно применяемое понятие, включающее фамилию, инициалы или имя и отчество, псевдоним, личное имя или прозвище в качестве фамилии), наименование организации, унифицированное заглавие произведения, обозначение документа, географическое название, иные сведения. Заголовок применяют при составлении записи на произведение одного, двух и трех авторов. Если авторов четыре и более, то заголовок не применяют, запись составляют под заглавием произведения.

При наличии двух и трех авторов указывают только имя первого автора или выделенного на книге каким-либо способом (цветом, шрифтом). Имена всех авторов приводят в библиографическом описании в сведениях об ответственности.

Основным заглавием является заглавие книги или статьи, а сведением, относящимся к заглавию - пояснение жанра, типа издания, например, сборник статей, учебное пособие и т.п.

Сведения об ответственности - это сведения о соавторах, переводчиках, редакторах и/или о той организации, которая принимает на себя ответственности за данную публикацию.

Сведения об издании включают качественную и количественную характеристику документа - переработанное, стереотипное, 2-е и т. п.

Выходные данные - это наименование города, издательства, где опубликована книга и года издания. Москва, Ленинград, Санкт-Петербург, Лондон, Париж и Нью-Йорк сокращаются (М., Л., СПб., L., P., N-Y.). Все остальные города пишутся полностью (Новосибирск, Киев). Названия издательств книг, опубликованных до 1917 года, пишутся полностью. Дата для книги означает год издания.

Объем - это количество страниц или страницы, на которых опубликована статья в журнале или сборнике.

В данном исследовании основное внимание сконцетрировано на обработке и структуризации затекстовых библиографических ссылок, оформление которых реглируется стандартом ГОСТ 5.0.7-2008 и имеет следующий общий вид:
\begin{itemize}
	\item заголовок;
	\item основное заглавие документа;
	\item общее обозначение материала;
	\item сведения, относящиеся к заглавию;
	\item сведения об ответственности;
	\item сведения об издании;
	\item выходные данные;
	\item физическую характеристику документа;
	\item сведения о местоположении объекта ссылки в документе (если ссылка на часть документа);
	\item сведения о серии;
	\item обозначение и порядковый номер тома или выпуска (для ссылок на публикации в многочастных или сериальных документах);
	\item сведения о документе, в котором опубликован объект ссылки;
	\item примечания;
	\item Международный стандартный номер.
\end{itemize}

\subsection{Стандарты представления библиографических данных}

В качестве примеров представления библиографических данных рассмотрим российский стандарт ГОСТ 7.0.5-2008 и два международных - АПА и АМА.

\subsubsection*{ГОСТ 7.0.5-2008}

Стандарт ГОСТ 7.0.5-2008 устанавливает общие требования и правила составления библиографической ссылки: основные виды, структуру, состав, расположение в документах.

Стандарт распространяется на библиографические ссылки, используемые в опубликованных и неопубликованных документах на любых носителях.

Стандарт предназначен для авторов, редакторов, издателей.

По месту расположения в документе различают библиографические ссылки:
\begin{itemize}
	\item внутритекстовые, помещенные в тексте документа;
	\item подстрочные, вынесенные из текста вниз полосы документа (в сноску);
	\item затекстовые, вынесенные за текст документа или его части (в выноску).
\end{itemize}

В данном исследовании будут рассматриваться затекстовые библиографические ссылки.

Затекстовая библиографическая ссылка может содержать следующие структурные элементы:
\begin{itemize}
	\item заголовок;
	\item основное заглавие документа;
	\item общее обозначение материала;
	\item сведения, относящиеся к заглавию;
	\item сведения об ответственности;
	\item сведения об издании;
	\item выходные данные;
	\item физическую характеристику документа;
	\item сведения о местоположении объекта ссылки в документе (если ссылка на часть документа);
	\item сведения о серии;
	\item обозначение и порядковый номер тома или выпуска (для ссылок на публикации в многочастных или сериальных документах);
	\item сведения о документе, в котором опубликован объект ссылки;
	\item примечания;
	\item Международный стандартный номер.
\end{itemize}

Примеры библиографических ссылок, оформленных согласно ГОСТ 5.0.8 - 2007:
\begin{itemize}
	\item Экономика и политика России и государств ближнего зарубежья : аналит. обзор, апр. 2007 / Рос. акад. наук, Ин-т мировой экономики и междунар. отношений. М., : ИМЭМО, 2007. 39 с.
	\item Валукин М. Е. Эволюция движений в мужском классическом танце. М. : ГИТИС, 2006. 251 с.
	\item Ковшиков В. А., Глухов В. П. Психолингвистика: теория речевой деятельности : учеб, пособие для студентов педвузов. М. : Астрель ; Тверь : ACT, 2006. 319 с. (Высшая школа).
	\item Содержание и технологии образования взрослых: проблема опережающего образования : сб. науч. тр. / Ин-т образования взрослых Рос. акад. образования ; под ред. А. Е. Марона. М. : ИОВ, 2007. 118 с.
	\item Ефимова Т. Н., Кусакин А. В. Охрана и рациональное использование болот в Республике Марий Эл // Проблемы региональной экологии. 2007. № 1. С. 80-86.
	\item Дальневосточный международный экономический форум (Хабаровск, 5-6 окт. 2006 г.) : материалы / Правительство Хабар, края. Хабаровск : Изд-во Тихоокеан. гос. ун-та, 2006. Т. 1-8.
	\item О внесении изменений в статью 30 закона Ненецкого автономного округа «О государственной службе Ненецкого автономного округа» : закон Ненец, авт. окр. от 19 мая 2006 г. № 721-ОЗ: принят Собр. депутатов Ненец, авт. окр. 12 мая 2006 г. // Няръяна вындер (Крас, тундровик) / Собр. депутатов Ненец, авт. окр. — 2006. — 24 мая.
	\item Об индивидуальной помощи в получении образования: (О содействии образованию) : федер. закон Федератив. Респ. Германия от 1 апр. 2001 г. // Образовательное законодательство зарубежных стран. — М., 2003. — Т. 3. — С. 422—464.
\end{itemize}

\subsubsection*{АПА}

АПА (APA, American Psychological Assosiation) - стандарт оформления библиографических ссылок, разработанный в Американской ассоциации психологов.

Стандарт АПА широко распространен и применяется в научных журналах, книгах и учебной литературе.

Данный стандарт был разработан и опубликован в 1929 году. В дальнейшем он пересматривался и дополнялся, новые редакции выходили в 1974, 1983, 1994, 2001 и 2009 годах.

Последняя редакция стандарта, опубликованная в 2009 году, использует ссылочную систему «автор-год издания» и содержит простую структуру со следующими структурными элементами:
\begin{itemize}
	\item авторы;
	\item год издания в круглых скобках;
	\item название статьи;
	\item название журнала;
	\item номер тома или выпуска;
	\item диапазон страниц, относящихся к статье;
	\item название издательства.
\end{itemize}

Авторы статьи и год издания пишутся через пробел и отделяются от названия статьи точкой. Название журнала отделяется от названия статьи точкой, номер тома или выпуска и диапазон страниц перечисляются через запятую. Название издательства отделяется точкой.

Примеры библиографических ссылок, использующих стандарт АПА:
\begin{itemize}
	\item Schmidt, F. L., Oh, I.-S. (2016). The crisis of confidence in research findings in psychology: Is lack of replication the real problem? Or is it something else? Archives of Scientific Psychology, 4, 32–37.
	\item Brown, B. (2010). The gifts of imperfection: Let go of who you think you're supposed to be and embrace who you are. Center City, MN: Hazelden.
	\item Singh, A. A., Hwahng, S. J., Chang, S. C., White, B. (2017). Affirmative counseling with trans/gender-variant people of color. In A. Singh, L. M. Dickey (Eds.), Affirmative counseling and psychological practice with transgender and gender nonconforming clients (pp. 41–68). Washington, DC: American Psychological Association.
	\item American Psychological Association. (n.d.). Divisions. Retrieved from http://www.apa.org/about/division/
\end{itemize}

\subsubsection*{АМА}

АМА (AMA, American Medical Assosiation) - стандарт оформления библиографических ссылок, разработанный организацией JAMA при поддержке Оксфордского университета.

Первая редакция стандарта была разработана и опубликована в 1962 году, а ее последняя, десятая редакция вышла в свет в 2007 году.

Стандарт АМА широко распространен и применяется в научных журналах, книгах и учебной литературе.

Последняя редакция стандарта, опубликованная в 2007 году, содержит структуру со следующими структурными элементами:
\begin{itemize}
	\item авторы;
	\item название статьи;
	\item название журнала;
	\item год издания;
	\item номер тома или выпуска;
	\item диапазон страниц, относящихся к статье.
\end{itemize}

Авторы статьи, название статьи, название журнала и год издания отделяются точкой и пробелом. Год издания и номер тома или выпуска отделяются точкой с запятой без пробела. Номер тома или выпуска и диапазон страниц отделяются двоеточием без пробела.

Примеры библиографических ссылок, использующих стандарт АМА:
\begin{itemize}
	\item Domingo J. Influence of cooking processes on the concentrations of toxic metals and various organic environmental pollutants in food: a review of the published literature. Crit Rev Food Sci Nutr. 2011;51(1):29-37.
	\item Hobbs S. Attitudes, practices, and beliefs of individuals consuming a raw food diet. Explore. 2005;1:272-277.
	\item Fujita A, Hashimoto Y, Nakahara K, Tanaka T, Okuda T, Koda M. Effects of a low calorie vegan diet on disease activity and general conditions in patients with rheumatoid arthritis. Rinsho Byori. 1999;47(6):554-560.
	\item Garcia A, Koebnick C, Hoffmann I, et al. Long-term strict raw food diet is associated with favourable plasma beta-carotene and low plasma lycopene concentrations in Germans. Br J Nutr. 2008;99(6):1293-1300.
	\item Koebnick C, Garcia A, Hoffmann I, et al. Long-term consumption of a raw food diet is associated with favorable serum LDL cholesterol and triglycerides but also with elevated plasma homocysteine and low serum HDL cholesterol in humans. J Nutr. 2005;135(10):2372-2378
\end{itemize}

\subsection{Машиночитаемые форматы библиографических данных}

\subsubsection*{RUSMARC}

Впервые программа MARC I была разработана Библиотекой Конгресса США в 1965-1966 годах с целью получения данных каталогизации в машиночитаемой форме. Аналогичная работа выполнялась в Великобритании Советом по Британской национальной каталогизации для обеспечения использования машиночитаемых данных при подготовке печатного издания Британской национальной библиографии — British National Bibliography (проект BNB MARC). На основе указанных разработок в 1968 г. начал создаваться коммуникативный англо-американский формат MARC (проект MARC II). Целями его создания стало обеспечение:

\begin{itemize}
	\item гибкости решения каталогизационных и других библиотечных задач,
	\item пригодности для национального библиографического описания любых видов документов и использования структуры записи в автоматизированных системах.
\end{itemize}

В 1971 году обобщённая версия MARC была принята в качестве международного стандарта ISO 2709.

Для преодоления несовместимости национальных форматов, ориентированных на национальные правила каталогизации, в 1977 году Международной федерацией библиотечных ассоциаций (ИФЛА) было выпущено издание «Универсальный формат MARC» (Universal MARC Format, UNIMARC). Его целью провозглашено « …содействие международному обмену данными в машиночитаемой форме между национальными библиографическими службами». Предполагалось, что этот формат должен стать посредником между любыми национальными версиями форматов MARC и, следовательно, обеспечивать конвертирование данных из национального формата в него, а из него — в другой национальный формат.

UNIMARC поддерживается международной организацией ИФЛА и использующийся в основном в Европе и Азии.

В своей второй редакции 1994 г. и с учетом дополнений последних лет формат UNIMARC включает поля, необходимые для описания таких видов документов, как текстовые документы монографические (прежде всего, современные книги), старопечатные издания, сериальные издания, нотные документы, графические материалы на непрозрачной основе, аудиоматериалы, видео- и проекционные материалы, электронные ресурсы, картографические материалы.

Поля формата UNIMARC можно подразделить на общие и специфические. Общие поля используются при описании любых видов документов, специфические — только при описании определенных видов. Специфические поля встречаются в блоках полей формата 0ХХ, 1ХХ, 2ХХ. В блоке 0ХХ имеются поля для записи уникальных международных идентифицирующих номеров документов (ISBN, ISSN, ISMN и т. д.). В блоке 1ХХ существуют поля кодированных данных отдельно для книг (105), сериальных изданий (110), видеоматериалов (115), графических материалов (116), электронных ресурсов (135). В блоке описательной информации 2ХХ специфическое поле 230 отражает область специфических сведений об электронных ресурсах.

UNIMARC включает достаточно большой перечень полей, однако даже этого перечня не хватает для описания специальных видов научной и технической литературы: диссертаций, отчетов по НИОКР, патентных, нормативно-технических документов и промышленных каталогов, причем не хватает именно специфических полей.

Формат UNIMARC разрабатывался на протяжении ряда лет, он постоянно совершенствуется и теперь, но очень медленно. Это связано с тем, что Постоянный комитет при ИФЛА, поддерживающий UNIMARC, малочислен и работает на общественных началах по принципу консенсуса, используя в основном переписку для взаимных консультаций. Одним из последних крупных изменений, внесенных в структуру формата Постоянным комитетом, является утверждение комплекса полей для описания электронных ресурсов, многие из которых имели статус предварительных еще в редакции формата 1987 г. Все перечисленные обстоятельства побудили разрабатывать свою версию формата, добавляя поля и подполя национального использования, что допускается данным международным стандартом. Кроме того, для большинства видов документов было решено разработать руководства по применению MARC-формата, которые должны были бы включать описания особенностей заполнении специфических и общих полей для каждого вида, а также содержать рекомендации по описанию в национальной версии формата типовых документов, относящихся к каждому виду, то есть была поставлена задача разработки образцов, или моделей, описания документов.

Формат RUSMARC официально признан Постоянным Комитетом IFLA по формату UNIMARC в качестве национальной адаптации последнего.
Формат RUSMARC зарегистрирован Комитетом по Z39.50 как формат, совместимый с этим протоколом.
Приказом Министра культуры  Российской Федерации № 45 от 27.01.1998 года формат RUSMARC признан обязательным при организации обмена данными для подведомственных библиотек.
Другой Приказ № 139 от 29.02.2000  года предписывает считать формат RUSMARC «обязательным при разработке и внедрении автоматизированных библиотечно-информационных систем для библиотек системы Министерства культуры Российской Федерации».

\subsubsection*{Bibtex}

BibTeX использует bib-файлы специального текстового формата для хранения списков библиографических записей. Каждая запись описывает ровно одну публикацию — статью, книгу, диссертацию, и т. д.

Bib-файлы можно использовать для хранения библиографических баз данных. Многие программы, работающие с библиографиями, (такие, как JabRef) и онлайн-сервисы цитирования (ADS, CiteULike) могут экспортировать ссылки в bib-формат.

Каждая запись выглядит следующим образом:

@ARTICLE\{tag, \par
	author = \{Список авторов\}, \par
	title = \{Название статьи\}, \par
	year = \{год\}, \par
	journal = \{Название журнала\} \par
\} \par

Здесь ARTICLE — тип записи («статья»), tag — метка-идентификатор записи (которая позволяет ссылаться в тексте), дальше список полей со значениями.

Ниже приведены наиболее распространенные типы публикаций, к которым может принадлежать определяемая запись:

\begin{itemize}
\item article - статья из журнала (необходимые поля: author, title, journal, year; дополнительные поля: volume, number, pages, month, note, key);
\item book - определённое издание книги (необходимые поля: author/editor, title, publisher, year; дополнительные поля: volume, series, address, edition, month, note, key, pages).
\end{itemize}

Каждая запись содержит некоторый список стандартных полей, перечисленных ниже в алфавитном порядке:

\begin{itemize}
	\item address: адрес издателя (обычно просто город, но может быть полным адресом для малоизвестных издателей);
	\item annote: аннотация для библиографической записи;
	\item author: имена авторов;
	\item booktitle: наименование книги, содержащей данную работу;
	\item chapter: номер главы;
	\item crossref: ключ кросс-ссылки (позволяет использовать другую библио-запись в качестве названия, например, сборника трудов);
	\item edition: издание (полная строка, например, «1-е, стереотипное»);
	\item editor: имена редакторов (оформление аналогично авторам);
	\item howpublished: способ публикации, если нестандартный;
	\item institution: институт, вовлечённый в публикацию, необязательно издатель;
	\item journal: название журнала, содержащего статью;
	\item month: месяц публикации (может содержать дату). Если не опубликовано — создания;
	\item note: заметки;
	\item number: номер журнала;
	\item organization: организатор конференции;
	\item pages: номера страниц, разделённые запятыми или двойным дефисом. Для книги — общее количество страниц;
	\item publisher: издатель;
	\item school: институт, в котором защищалась диссертация;
	\item series: серия, в которой вышла книга;
	\item title: название работы;
	\item type: тип отчёта, например «Заметки исследователя»;
	\item volume: том журнала или книги;
	\item year: год публикации (если не опубликовано — создания).
\end{itemize}

Дополнительно, каждая запись содержит ключевое поле, которое служит для цитирования или кросс-ссылок на эту запись. Это поле должно быть уникальным (в рамках использующей работы) и непустым. Это поле не имеет названия, не является частью других полей и идёт первым по-порядку.

Обычно BibTeX формирует вывод в формате TeX или LaTeX, но существуют и стилевые файлы для генерации формата HTML.

Многие журналы или издательства, которые принимают публикации в формате LaTeX, снабжают также библиографическими стилями для авторов. Это удостоверяет, что стиль оформления библиографии также будет соответствовать требованиям издателя с минимальными усилиями.

\subsubsection*{RIS}

RIS - это стандарт представления библиографических данных в текстовом виде для обеспечения обмена этими данными между другими пользователями и программами.

Библиографическая запись, оформленная в соответствии со стандартом RIS представляет собой текстовый файл, содержащий строки следующего вида: двухбуквенный код структурного элемента библиографической записи и его значение, разделенные чертой с одним пробелом с каждой стороны.

В соответствии с описанием стандарта, каждая строка должна оканчиваться на знаки возврата каретки и переноса строки.

В одном файле может быть описано несколько библиографических записей, которые должны быть разделены двухбуквенным кодом ER.

Каждая запись выглядит следующим образом:

TY - JOUR \par
AU - Автор \par
TI - Название публикации \par
PY - Дата публикации \par
T2 - Название журнала \par
ER - 

Наиболее распространенные двухбуквенные коды структурных элементов:

\begin{itemize}
	\item TY - тип публикации;
	\item AU - автор;
	\item PY - дата публикации;
	\item TI - название публикации;
	\item T2 - название статьи;
	\item SP - номер первой страницы публикации;
	\item EP - номер последней страницы публикации;
	\item VL - номер тома;
	\item ER - конец описания библиографической записи;
	\item IS - номер выпуска;
	\item PB - издатель;
	\item UR - адрес в сети Интернет.
\end{itemize}

Наиболее распространенные типы публикаций:

\begin{itemize}
	\item BOOK - книга;
	\item CONF - тезисы конференции;
	\item ELEC - страница в сети Интернет;
	\item JOUR - журнал;
	\item THES - диссертация.
\end{itemize}

\section{Анализ основных параметров, достоинств и недостатков современных информационных систем для обработки и структуризации библиографических данных}

В данном разделе содержится обзор информационных систем для обработки и структуризации библиографических данных.

Bibus bibliographic database — система управления библиографиями и библиографическими ссылками: поиск, редактирование и сортировка библиографических сведений.
Ключевые особенности: свободное ПО, иерархическая организация ссылок, многопользовательское применение, экспорт в LibreOffice, OpenOffice и Microsoft Word, поддержка Unicode.
Недостатки: малое число импортируемых и экспортируемых форматов, отсутствие коррекции ошибок, нет поддержки русского языка, отсутствие возможностей по созданию собственных форматов импорта и экспорта, ручной ввод структурных элементов библиографических данных, 

jabRef — система управления библиографическими данными.
Ключевые особенности: открытое ПО, поддержка импорта RIS, Medline/Pubmed (xml), Refer/Endnote, INSPEC, BibTeXML, CSA, ISI Web of Science, SilverPlatter, Scifinder, OVID, Biblioscape, Sixpack, JStor и RIS, поддержка экспорта HTML, Docbook, BibTeXML, MODS, RTF, Refer/Endnote и LibreOffice, поддержка собственных форматов импорта и экспорта.
Недостатки: отсутствие русскоязычной документации, ограниченность форматов импорта и экспорта текстовыми представлениями, однопользовательский режим работы, отсутствие коррекции ошибок, ручной ввод структурных элементов библиографических данных.

МАРК-SQL — отечественная автоматизированная информационно-библиотечная система.
Ключевые особенности: поддержка различных СУБД, расширяемость, масштабируемость, дружественный интерфейс.
Недостатки: малое число поддерживаемых форматов (MARC21 и RUSMARC), отсутствие автоматизированных импорта и экспорта, зависимость от платформы Windows, платная лицензия, отсутствие коррекции ошибок, отсутствие возможностей по созданию собственных форматов импорта и экспорта, ручной ввод структурных элементов библиографических данных.

МАРК Cloud — отечественная автоматизированная информационно-библиотечная система нового поколения.
Ключевые особенности: каталогизация, электронная библиотека, учет пользователей, личный кабинет, читательские заказы и процессы информационного обслуживания, облачные вычисления.
Недостатки: малое число поддерживаемых форматов (MARC21 и RUSMARC), отсутствие автоматизированных импорта и экспорта, платная лицензия, отсутствие коррекции ошибок, отсутствие возможностей по созданию собственных форматов импорта и экспорта, ручной ввод структурных элементов библиографических данных.

Фолиант — система автоматизации библиотечных процессов.
Ключевые особенности: возможность использования в библиотеках любого типа, дружественный пользовательский интерфейс, модульная организация системы.
Недостатки: малое число поддерживаемых форматов импорта (RUSMARC, USMARC для экспорта и импорта, ГОСТ 7.1-2003, ГОСТ 7.80-2000, ГОСТ 7.82-2001, ГОСТ 7.51-98, ГОСТ 7.12-93, ГОСТ 7.11-78, ГОСТ 7.83-2001, ГОСТ 7.59-90 для экспорта), отсутствие коррекции ошибок, отсутствие возможностей по созданию собственных форматов импорта и экспорта, платная лицензия, ручной ввод структурных элементов библиографических данных.

EasyBib — веб-приложение формирования библиографических ссылок по ISBN.
Ключевые особенности: поиск источника по ISBN и формирование библиографической ссылки, веб-приложение, не требующее установки на компьютер, бесплатность использования.
Недостатки: отсутствие хранения результатов, отсутствие импорта, экспорт в малое число текстовых форматов (MLA7, MLA8, APA, Chicago), ручной ввод структурных элементов библиографических данных.

\section*{Выводы к главе 1}
\addcontentsline{toc}{section}{Выводы к главе 1}

Обзор принципов построения информационных систем для обработки и структуризации библиографических данных позволил сформулировать определение информационной системы, способной решить поставленную задачу обработки и структуризации библиографических данных, которая будет рассматриваться в данном исследовании. Такая информационная система является локальной, клиент-серверной, автоматизированной, решающей и персональной с одним оператором.

ИС для обработки и структуризации библиографических данных должна соответствовать следующим принципам: принципу системности, принципу открытости, принципу совместимости, принципу унификации, принципу эффективности.

База данных для хранения библиографической информации является важной частью информационной системы и должна соответствовать следующим принципам: высокое быстродействие, простота обновления данных, независимость данных, совместное использование данных многими пользователями, безопасность данных, адекватность отображения данных в соответствии с предметной областью, простой пользовательский интерфейс.

Обзор российских и международных стандартов оформления библиографических данных позволил ознакомиться с формами информационного представления библиографических записей. В качестве основного стандарта оформления данных для обработки и структуризации был выбран ГОСТ 5.0.7-2008, являющийся основным российским стандартом оформления библиографических записей.

Рассмотрение машиночитаемых стандартов представления библиографических данных позволило выделить ключевые структурные элементы библиографических записей для информационного обмена с помощью компьютера, что является необходимым для решении задачи структуризации библиографических данных в информационных системах.

Были выделены следующие структурные элементы: имена авторов, название публикации, название издания, местоположение и название издательства, номер тома или выпуска, год издания и диапазон страниц.

Обзор существующих информационных систем для обработки и структуризации библиографических данных показал, что существующие решения обладают одним или несколькими из следующих принципиальных недостатков: ручной ввод структурных элементов библиографических записей для дальнейшего структурирования, точное совпадение библиографических данных с определенным стандартом, плохая поддержка русского языка или полное ее отсутствие.

Таким образом обзор современного состояния информационных систем для обработки и структуризации библиографических данных показал, что задача диссертации, заключающаяся в исследовании и разработке методики и алгоритма обработки и структуризации библиографических данных, является актуальной и найдет свое применение в существующих и принципиально новых информационных системах.
